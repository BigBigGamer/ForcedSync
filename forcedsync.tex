\input{text/pre}
\usepackage{mwe}

\begin{document}

\def\labauthors{Виноградов И.Д., Шиков А.П.}
\def\labgroup{430}
\def\labnumber{1}
\def\labtheme{Вынужденная синхронизация}
\input{text/titlepage}

\newpage
\section*{Эксперимент}
\textbf{Оборудование}
\begin{enumerate}
	\item Синхронизируемый автогенератор.
	\item Генератор внешней силы (с регулировкой частоты и амплитуды ).
	\item Осциллограф.
\end{enumerate}

\subsection*{Схема лабораторной установки}

\begin{figure}[h!]
	\centering
	\includegraphics[width = .7\linewidth]{img/scheme.png}
	\caption{Схема лабораторной установки}
	\label{fig:2}
\end{figure}

\subsection*{Мягкий режим}
\subsubsection*{Изучение явления захватывания при мягком режиме возбуждения автогенератора}
Был выставлен мягкий режим возбуждения генератора, для этого на управляющую сетку было подано напряжение 1.3 В.
Амплитуда колебаний составляла $U_{amp} = 9$ В , на частоте $f_0 = 426.7$ кГц.

При поданом внешнем воздействии, снималась зависимость амплитуды колебаний от частоты внешнего сигнала (АЧХ, см рис. \ref{fig:3}).
Зависимость снималась только для синхронного режима, измерения прекращались, при переходе в режим биений.

\begin{figure}[H]
	\centering
	\includegraphics[width = .99\linewidth]{graphs/soft1.pdf}
	\caption{АЧХ для мягкого режима автогенератора}
	\label{fig:3}
\end{figure}

Кривые для значени апмплитуды внешней силы $U_{out} = 1.3$ В и $U_{out} = 1$ В соответствуют сильному сигналу, $U_{out}
= 0.3$ В и $U_{out} = 0.6$ В - слабому. Тип сигнала определялся по осциллограммам колебаний.

\subsubsection*{Зависимость границ полосы удержания и захвата от амплитуды внешнего воздействия}

% \begin{figure}[H]
% 	\centering
% 	\includegraphics[width = .4\linewidth]{img/sync.pdf}
% 	\label{fig:sync}
% \end{figure}

При тех же параматерах автогенератора снималась зависимость границ полосы удержания и полосы захвата от амплитуды внешнего воздействия. 

Для измерения полосы захвата, частота менялась таким образом,чтобы переход происходил из области биений в область
колебаний. Для полосы удержания частота менялась из области колебаний в область биений.

\begin{figure}[H]
	\centering
	\includegraphics[width = .85\linewidth]{graphs/sync.png}
	\caption{Зависимость границ полосы синхронизации от внешней амплитуды}
	\label{fig:4}
\end{figure}

Как видно из рис. \ref{fig:4}. полоса захвата всегда меньше полосы удержания. Также ширина полосы синхронизации и
разность между полосой удержания и полосой захвата увеличивается с увеличением амплитуды внешнего воздействия.

Также была измерена зависимость амплитуды колебаний на (правой) границе полосы синхронизации.

\begin{figure}[H]
	\centering
	\includegraphics[width = .85\linewidth]{graphs/amps1.png}
	\caption{Амплитуда на правой границе полосы синхронизации}
	\label{fig:5}
\end{figure}

\begin{figure}[H]
	\begin{minipage}{.45\linewidth}
		\centering
		\includegraphics[width=\linewidth]{img/1s.jpg}
		\caption*{а}
	\end{minipage}
	\begin{minipage}{.45\linewidth}
		\centering
		\includegraphics[width=\linewidth]{img/0s.jpg}
		\caption*{б}
	\end{minipage}
\end{figure}

\begin{figure}[H]
	\begin{minipage}{.45\linewidth}
		\centering
		\includegraphics[width=\linewidth]{img/2s.jpg}
		\caption*{в}
	\end{minipage}
	\begin{minipage}{.45\linewidth}
		\centering
		\includegraphics[width=\linewidth]{img/3s.jpg}
		\caption*{г}
	\end{minipage}
	\caption{Фигура Лиссажу и осциллограмма в режиме синхронизации (а, б) и биений (в, г)}
	\label{fig:6}
\end{figure}


\begin{figure}[H]
	\begin{minipage}{.45\linewidth}
		\centering
		\includegraphics[width=\linewidth]{img/5s.jpg}
		\caption*{Сильный сигнал}
	\end{minipage}
	\begin{minipage}{.45\linewidth}
		\centering
		\includegraphics[width=\linewidth]{img/6s.jpg}
		\caption*{Слабый сигнал}
	\end{minipage}
	\caption{Осциллограмма режима биений в окрестностях границы полосы синхронизации}
	\label{fig:7}
\end{figure}

\subsection*{Жесткий режим}
\subsubsection*{АЧХ амплитуд внешнего сигнала}

Для достижения жесткого режима на сетке лампы выставлено смещение $U =  2.95$ В.

При увеличении обратной связи, колебания (без внешнего воздействия) появлялись при значении $P_1 = 75$, а при обратном
ходе пропадали при $P_2 = 35$. Измерения проводились при $P_2 < P <P_1, P = 50$. Построенные графики приведены на рис. \ref{fig:hard}. 

\begin{figure}[h!]
	\centering
	\includegraphics[width = .95\linewidth]{graphs/hard.pdf}
	\caption{АЧХ жесткого режима}
	\label{fig:hard}
\end{figure}

При амплитудах внешнего воздействия $U_{out} = 700$ мВ и $U_{out} = 500$ мВ наблюдается синхронизация. Кривые,
полученные при $U_{out} = 400$ мВ и $U_{out} = 350$ мВ соответствуют резонансу.


\end{document}